\input texinfo      @c -*-texinfo-*-
@c Copyright (c) 1992 - 1994  Heinz W. Werntges
@c Copyright (c) 1998 - 2000  Martin Kroeker
@c %**start of header
@setfilename hp2xx.info
@settitle HP2XX, A HP-GL Converter
@setchapternewpage odd
@c @smallbook
@c @cropmarks
@c %**end of header

@finalout
@syncodeindex ky cp

@tex
% \def\$#1${{#1}}  % Kludge: collect RCS revision info without $...$
\xdef\manvers{$Revision$}  % For use in headers, footers too
@end tex

@c HP2XX CHANGE LOG:
@c 92/06/11  V 0.90  HWW  Originating; derived from GDB.info V 4.04
@c 92/07/14  V 0.91  HWW  Various additions
@c 92/10/21  V 0.92  HWW  Update on "LT;" which is now supported,
@c                        Acknowledgement of RMS's comments on the manual
@c 92/12/13  V 1.00  HWW  Cleanup work; mode "cad"; flag -t; acknowledgement
@c 93/01/09  V 1.01  HWW  Option -S; long options; acknowledgement; install
@c 93/04/14  V 1.02  HWW  Option -S4; UC support
@c 93/09/02  V 1.03  HWW  EA support; RGIP & HPGL modes; DJ5x0: data compression
@c 93/11/23  V 1.04a HWW  Typos fixed
@c 94/02/15  V 1.04b HWW  Option -C added
@c 99/08/01  V 1.10  MK   Updated for 3.30 (new options, new HPGL commands)
@c 99/12/01  V 1.11  MK   Updated for 3.31 (HPGL commands,acknowledgements)
@c 00/02/06  V 1.12  MK   Added notes on relation between -c/-s and PC/PW
@c                        and on auto-generation of ppm files in pbm mode
@c 00/09/20  V 1.2   MK   fixed INFO-DIR-ENTRY, updates for 3.4.0
@c THIS MANUAL REQUIRES TEXINFO-2 macros and info-makers to format properly.

@ifinfo
@format
START-INFO-DIR-ENTRY
* HP2XX:(hp2xx).  Conversion of HP-GL graphics into raster and
                  vector graphics formats
END-INFO-DIR-ENTRY
@end format
@end ifinfo
@c
@ifinfo
This file documents the HP-GL converter HP2XX

@c !!set edition, date, version
This is Edition 1.2, September 2000,
of @cite{Using HP2XX: A HP-GL Converter}.

Copyright (C) 1998 - 2000 Martin Kroeker
Copyright (C) 1992 - 1994 Heinz W. Werntges

Permission is granted to make and distribute verbatim copies of
this manual provided the copyright notice and this permission notice
are preserved on all copies.

@ignore
Permission is granted to process this file through TeX and print the
results, provided the printed document carries copying permission
notice identical to this one except for the removal of this paragraph
(this paragraph not being relevant to the printed manual).

@end ignore
Permission is granted to copy and distribute modified versions of this
manual under the conditions for verbatim copying, provided also that the
accompanying file named COPYING which contains the ``GNU General Public
License'' is included exactly as in the original, and provided that the
entire resulting derived work is distributed under the terms of a permission
notice identical to this one.

Permission is granted to copy and distribute translations of this manual
into another language, under the above conditions for modified versions,
except that the abovementioned file COPYING containing the
``GNU General Public License'' may be included in a translation approved
by the Free Software Foundation instead of in the original English.
@end ifinfo

@titlepage
@title Using HP2XX
@subtitle A HP-GL Converter
@c @subtitle on XXXX Systems
@sp 1
@c !!set edition, date, version
@subtitle Edition 1.2, for HP2XX version 3.4.x
@subtitle September 2000
@author by Martin Kroeker (previously by Heinz W. Werntges)
@page
@tex
{\parskip=0pt
\hfill mk\@daveg.com\par
\hfill {\it Using HP2XX}, \manvers\par
\hfill \TeX{}info \texinfoversion\par
}
@end tex

@vskip 0pt plus 1filll
Copyright @copyright{} 1998 - 2000 Martin Kroeker
Copyright @copyright{} 1992 - 1994 Heinz W. Werntges

Permission is granted to make and distribute verbatim copies of
this manual provided the copyright notice and this permission notice
are preserved on all copies.

Permission is granted to copy and distribute modified versions of this
manual under the conditions for verbatim copying, provided also that the
accompanying file named COPYING which contains the ``GNU General Public
License'' is included exactly as in the original, and provided that the
entire resulting derived work is distributed under the terms of a permission
notice identical to this one.

Permission is granted to copy and distribute translations of this manual
into another language, under the above conditions for modified versions,
except that the abovementioned file COPYING containing the
``GNU General Public License'' may be included in a translation approved
by the Free Software Foundation instead of in the original English.
@end titlepage
@page

@ifinfo
@node Top, Introduction, (dir), (dir)
@comment  node-name,  next,  previous,  up
@top HP2XX, a HP-GL Converter

This file describes HP2XX, a converter of HP-GL plotter data into
some vector and raster formats

@c !!set edition, date, version
This is Edition 1.2, September 2000, for HP2XX Version 3.4.x
@end ifinfo


@menu
* Introduction::
* Basics::
* Advanced subjects::
* Installation and modification notes::
* Appendix A::                          Known HP-GL commands
* Appendix B::                          Option summary
* Appendix C::                          Acknowledgement, Copyright notice
@end menu

@node Introduction, Basics, Top, Top
@comment  node-name,  next,  previous,  up
@chapter Introduction

@menu
* Invoking hp2xx::          How to run hp2xx
* HP2XX for the impatient:: Learning to use hp2xx by examples
@end menu

The @code{hp2xx} program is a versatile tool to convert vector-oriented
graphics data given in Hewlett-Packard's HP-GL plotter language
into a variety of popular both vector- and raster-oriented graphics formats.

The various supported output formats include Encapsulated PostScript (EPS),
PCX, IMG, and several formats intended to facilitate the generation of
graphics within @TeX{} documents.
In addition, @code{hp2xx} output is printable on the HP Laserjet/Deskjet
printer series, and it may be used as a HP-GL previewer on many
platforms, e.g. X11 and DOS (VGA).

@code{hp2xx} first converts all HP-GL data into pure vectors and buffers
them internally. It then converts these vectors into a specified output
format (vector modes), or rasterizes them (raster modes) on an internal
bitmap. In raster modes, @code{hp2xx} then translates the bitmap into the
output format.



@node Invoking hp2xx, HP2XX for the impatient, , Introduction
@comment  node-name,  next,  previous,  up
@section Invoking @code{hp2xx}

The format of the @code{hp2xx} command is:

@example
hp2xx [@var{options}] [@var{input-file/s}]
@end example

It follows the UNIX System V tradition of a filter, i. e., options begin
with @samp{-}, followed by a single letter and an optional parameter.
Options must appear immediately behind the program name and before the input
file name(s) (if specified). If no input file is given, @code{hp2xx} reads
from @code{stdin}. In addition to this traditional option handling,
@code{hp2xx} also supports GNU-style long options and option/non-option
permutation (@pxref{Appendix B}). However, throughout this manual
all examples will only display short options.

@code{hp2xx} writes to the output file whose name can be specified
by option @samp{-f}. Without option @samp{-f}, @code{hp2xx} generates
output file names from the input names and the selected mode
(see option @samp{-m}). @code{hp2xx} writes to @code{stdout} if you supply
a dash as output file name like in @samp{-f-}.



@node HP2XX for the impatient, , Invoking hp2xx, Introduction
@comment  node-name,  next,  previous,  up
@section @code{hp2xx} for the impatient

This section is intended to give those of you a quick-start who are
quite familiar with traditional UNIX-style programs and with HP-GL and
other graphics formats.
The following examples will give you a good idea of @code{hp2xx}'s
functionality. @xref{Appendix B}, the Option Summary, for further details.


@example
hp2xx foo.hp
@end example

Preview of HP-GL graphics in file @file{foo.hp}. The picture will
fit into a square of 200 mm width, assuming that your output device
(screen) features 75 DPI resolution (default).

@example
hp2xx -q -d86 -h160 -w220  foo.hp bar.hp
@end example

Multiple-file preview. Option @samp{-q} puts @code{hp2xx} into "quiet" mode.
The picture will fit into a rectangle of 220 mm width and 160 mm height,
assuming a 86 DPI resolution of the output device (screen).

@example
hp2xx -t -c12340567 -p12230412 foo.hp
@end example

Preview, size according to original HP-GL data (as on a plotter),
with different pen colors and sizes. Color and width according to:
@display
  @code{Pen # : Color code     / Size (pixel)}
  @code{-------------------------------------}
  @code{  1   : 1 (black)      /    1}
  @code{  2   : 2 (red)        /    2}
  @code{  3   : 3 (green)      /    2}
  @code{  4   : 4 (blue)       /    3}
  @code{  5   : 0 (background) /    0}
  @code{  6   : 5 (cyan)       /    4}
  @code{  7   : 6 (magenta)    /    1}
  @code{  8   : 7 (yellow)     /    2}
@end display

@example
hp2xx -m eps -l a.log -h100 -w150 -p542  foo.hp bar.hp
@end example
Encapsulated Postscript mode. Files @file{foo.eps} and @file{bar.eps}
will be created. The diagnostic output will be written into @file{a.log},
so @code{hp2xx} works quietly. Both EPS pictures will fit into a
rectangle of size 150 x 200 mm. The size (width) of
pen 1: 0.5 mm, pen 2: 0.4 mm, pen 3: 0.2 mm, pen 4 ... pen 8: 0.1 mm.

@example
unix% cat foo.hp | hp2xx -m pcl -o30 -O50 -i -F -f- | lpr -P ljet
@end example

In this generic UNIX example, @code{hp2xx} reads HP-GL code from @code{stdin},
converts it to HP-PCL which is suitable for direct output on any HP Laserjet
printer, and pipes the output via @code{stdout} into the appropriate printer queue.
Option @samp{-f-} forces @code{hp2xx} to write to @code{stdout} instead of
a file, @samp{-i} initializes the printer before the output,
@samp{-F} sends a FormFeed at the end of output. There will be (additional)
30 mm left and 50 mm top margins. 75 DPI are assumed per default.

@example
hp2xx -m pcx -f foo3.pcx -d300 -h80 -w150 -r90 -P2:4  foo.hp
@end example
PCX mode. Output goes into file @file{foo3.pcx}. A limiting rectangle
of 150 x 80 mm at 300 DPI is assumed. The picture will be rotated by
90 degrees. Only pages 2 to 4 of the multi-page HP-GL source is used (each
occurrence of HP-GL code @code{PG;} increments the internal page counter).



@node Basics, Advanced subjects, Introduction, Top
@comment  node-name,  next,  previous,  up
@chapter Basics

@menu
* Modes of hp2xx::           Selection of the output format
* Sizing your output::       Specifying width, height, page offsets etc.
* Pen sizes and colors::     How to emulate pens of various thickness and color
* Selecting a page (range):: Converting multi-page HP-GL files
* Vector formats::           Further details
* Raster formats::           Controlling the resolution (DPI) etc.
* Printer formats::          Options and restrictions
* Preview::                  Some reminders for the unwary
* Misc. options::            Other bells and whistles
@end menu

This chapter provides you with almost anything you'll need for successful
@code{hp2xx} applications. You'll probably soon operate @code{hp2xx}
by solely consulting the option summary (@pxref{Appendix B}) or just
by calling @code{hp2xx} without any parameters to obtain its built-in
option summary.


@node Modes of hp2xx, Sizing your output, , Basics
@comment  node-name,  next,  previous,  up
@section Modes of @code{hp2xx}

The mode switch @samp{-m @code{string}} tells @code{hp2xx} about the mode it
should use to generate output, i.e., the desired output format. @code{hp2xx}
can run in three different groups of modes: Generating vector graphics,
raster graphics, or "preview mode", i.e., displaying the graphics.
Preview is the default; @pxref{Appendix B} for a list of all modes.



@node Sizing your output, Pen sizes and colors, Modes of hp2xx, Basics
@comment  node-name,  next,  previous,  up
@section Sizing your output

NOTE: The basic unit length within @code{hp2xx} is @code{mm} (millimeter).
This unit is always assumed except where noted otherwise.

In contrast to a real HP-GL plotter, @code{hp2xx} lets you decide freely
about the size of the picture. While preserving the picture's aspect ratio,
@code{hp2xx} will fit the picture into a window of width @var{w} and
height @var{h}. By default, @var{w} = @var{h} = 200 (mm). Set these basic
sizes using options @samp{-w @var{w}} and @samp{-h @var{h}}.

Sometimes you'll want to change the aspect factor of a picture, e.g., to
spread out a square picture into landscape. Option @samp{-a @var{af}} is used
for this. @var{af} > 1 increases x/y ratio, 0 < @var{af} < 1 decreases x/y.

Example: Let's assume your picture covers a native coordinate range of
100...900 plotter units in x direction and 200...600 in y direction.
Thus, its width is double its height.
Using defaults, @code{hp2xx} will create a picture of size 200 x 100 mm,
while options @samp{-w @var{100} -h @var{40}} will lead to a picture of
size 80 x 40 mm, and @samp{-w @var{100} -h @var{40} -a @var{0.5}} results
in a 40 x 40 mm picture.

Alternatively, ignore explicit size control and rely on the true
HP-GL coordinates (and therefore: sizes) of the given input file(s).
Flag @samp{-t} inhibits @samp{-a -h -w} and lets @code{hp2xx} use true
HP-GL sizes, based on the assumption that 1 HP unit = 1/40 mm.

Some modes of @code{hp2xx} support page offsets, i.e., left
and upper margins added to the picture, probably in addition to some
hard margins which cannot be avoided. Currently, these modes are
@var{eps}, @var{pcl}, and @var{pre}. The left margin (offset) is modified
with option @samp{-o @var{off_left}}, while the upper margin can be
controlled via @samp{-O @var{off_upper}}. @var{off_left} and @var{off_upper}
are specified in mm.

Option @samp{-C} (center the picture) works in combination with options
@samp{-o -O} by modifying the offsets in such a way that the resulting
picture is centered within the frame defined by options @samp{-w -h}. Example:
Assume that the picture is a square, and @samp{-w @var{100} -h @var{40}}
are specified. A 40 mm square with an additional left margin of
(100-40)/2 = 30 mm will be produced if option @samp{-C} is given.
If the actual width/height ration of the picture exactly matches
the ratio defined by @samp{-w -h}, option @samp{-C} has no effect.



@node Pen sizes and colors, Selecting a page (range), Sizing your output, Basics
@comment  node-name,  next,  previous,  up
@section Pen sizes and colors

Imagine a plotter with a pen carousel, e.g., like the model HP7550A.
The carousel carries a (small) number of pens. Their colors and tip
thicknesses (sizes) are selected by a human operator, while the plotter
only receives commands like "Now use pen number 5". If you don't provide a
pen, the plotter will move and ``draw'' without this pen if its number
is selected.

@code{hp2xx} emulates a carousel of up to 8 pens of various colors and sizes.
By default, all pens are present, have @var{foreground} color
(typically @var{black}), and their tip thicknesses are one unit
(here: 1/10 @code{mm} for vector modes, 1 pixel for raster modes).
HPGL/2 commands NP,PC,PW may override this for up to 255 freely configurable
pens.

Pen colors and sizes are represented by digits to allow for a compact
option list. There are 8 colors including @var{background} (usually white).
@xref{Appendix B}, the Option Summary, for a list of all colors. E.g., color 3 is green,
and color 7 means yellow. Permitted pen sizes are 0 ... 9 units.
WARNING: In raster modes, all pen sizes are approximated by double, triple,
etc. width, and lines wider than 4 will probably not give useful results.

Options @samp{-c @var{c-string}} and @samp{-s @var{s-string}} tell
@code{hp2xx} about the pens to be placed in the carousel.
@var{c-string} and @var{s-string} are strings of 1 to 8 digits,
corresponding to special choices of pen 1 to 8.
Defaults are @var{c-string} = @var{s-string} = @samp{11111111} unless the
hpgl file contains corresponding PC and/or PW commands. Specifying @samp{-c}
or @samp{-s} overrides the equivalent HPGL/2 command.
If you specify less than 8 pens, the remaining pens keep their defaults.

Examples:
@example
hp2xx -p13 foo.hp
@end example
Show a preview of @samp{foo.hp}, drawing all lines with pen #2 three
pixels wide instead of default 1 pixel, which applies to all other pens.

@example
hp2xx -c12740 -p12230412 foo.hp
@end example
Here, pen #5 is ``removed''. Pens #1 and #7 keep their default sizes,
all others are set to various sizes.
Pen #2 is red, #3 is yellow, and #4 is blue, while all other
pens keep @samp{foreground} color, e.g., black.

If your HP-GL file contains the PC and/or PW commands introduced with
HP-GL/2, these are interpreted to allow up to 255 pens. The special case
where a pen is redefined to take on different colors is currently only
supported in PostScript output mode. In all other modes, the last PC
command encountered determines that pens' color for the entire plot.


@node Selecting a page (range), Vector formats, Pen sizes and colors, Basics
@comment  node-name,  next,  previous,  up
@section Selecting a page

There is a HP-GL command named @samp{PG;} which amounts to a FormFeed.
Thus, there are multi-page HP-GL sources. While @code{hp2xx} was
designed for just one output picture per input file, there is a
simple way to cope also with multi-page sources:

@code{hp2xx} keeps track of the number of encountered @samp{PG;} commands.
All code up to the next (if any) @samp{PG;} command is considered a
single page. Pages are counted, starting at 1. You can ask @code{hp2xx}
to ignore all HP-GL commands other than on page @var{n} with option
@samp{-P @var{n}}, effectively filtering out any one-page graphics.
Sometimes, converting a whole page range makes sense, too. Therefore,
@code{hp2xx} also accepts page ranges via @samp{-P @var{n1}:@var{n2}}.
The default is @samp{-P @var{0}} which selects all pages.

WARNING: Some HP-GL sources may start with a @samp{PG;} so the first
page of your graphics may be 2 instead of 1. Look for the number of
encountered pages in the diagnostic output if you miss the expected page!
If the detected coordinate range shows unreasonable numbers like 1e10,
you'll be probably looking at an empty page.




@node Vector formats, Raster formats, Selecting a page (range), Basics
@comment  node-name,  next,  previous,  up
@section Vector formats

All HP-GL graphics are decomposed by @code{hp2xx} into elementary
move and draw commands. Selecting a vector mode essentially defines the
conversion rules of such commands into specific formats.

The most popular and versatile vector format currently is
Encapsulated PostScript (EPS). (In fact, it is much more than just a
vector graphics format, but @code{hp2xx} uses only EPS's line drawing
features.) Many programs allow importation of EPS files, and PostScript
gives excellent printing results, so @samp{-m @code{eps}} is highly
recommended. Recently, Adobe's Portable Document Format (PDF) has mostly taken
its place for formatted online documentation. For Web-based applications,
the Scalable Vector Graphics (SVG) format, a W3C standard, holds some promise
as a non-proprietary, XML-based format for vector images, not only on web pages.

Unix users may also want to use either @code{Gnuplot} or especially @code{XFig}
to annotate their plots. The native formats of these programs are directly
supported via the @samp{-m @code{gpt}} and @samp{-m @code{fig}} options.

Currently, all other supported vector formats represent various compromises
to persuade @TeX{}@ or La@TeX{}@ into the generation of graphics.
@xref{TeX formats}, if you are specially interested in @TeX{}.



@node Raster formats, Printer formats, Vector formats, Basics
@comment  node-name,  next,  previous,  up
@section Raster formats

Raster graphics are probably the most widely used graphics by now.
Many publishing programs accept rasterized graphics. It's likely you'll
use @code{hp2xx} primarily in some raster mode.

In addition to vector modes, all raster modes need the desired resolution
of an assumed underlying pixel grid to plot on, i. e., the number of
pixels per unit length within that grid.
A traditional measure is the number of ``dots per inch'' (DPI).
@code{hp2xx} makes an exception from its usual unit length @code{mm} and
lets you specify the traditional DPI values.
Option @samp{-d @var{num}} affects both x and y direction, @var{num} being
the DPI value (an integer). if @samp{-D @var{num_y}} is also specified,
@var{num_y} will override the @var{num} DPI value, but only for the y
direction.

There are plenty of raster formats on various platforms, much more
than @code{hp2xx} will ever handle. The supported raster formats
IMG, JPG, PBM/PPM, PCL, PCX, PNG and TIFF were chosen for their widespread use, their simplicity,
for actual demand, and for accessibility of specifications.
If your desired format is not supported, look for a converter. E.g., the
Portable Bitmap (PBM) project and more recently, the ImageMagick package 
offer quite a variety of such converters.

Please note that the preview mode (which does not create any output file)
is a special raster mode. Instead of going into some output file,
the internal bitmap is transferred into display memory.
Therefore, the above considerations apply also to preview mode.




@node Printer formats, Preview, Raster formats, Basics
@comment  node-name,  next,  previous,  up
@section Printer formats

Currently, there is only one well-tested printer format (not counting @samp{eps}, which
is printable on PostScript printers): @samp{pcl}, which stands for HP PCL
Level 3. Essentially it is a raster format, but it comes with a few
restrictions and additional options which correspond to printer properties.
It prints on HP Laserjet and HP Deskjet series printers and compatibles.

The restriction concerns the resolution (DPI) during rasterization. Due to
printer hardware limitations, only @samp{-d 75}, @samp{-d 100}, @samp{-d 150},
and @samp{-d 300} are recommended (@code{hp2xx} will emit a warning for other
values, but will try to create the desired output); option @samp{-D} must not 
be used. Some recent printer models do support 600, 720 or even higher
resolutions, but even for those the requirements both in terms of cpu load
and file size may be prohibitive.

There are two flags which may be useful if the output goes directly
to a printer: Option @samp{-i} initializes the printer and tries to select
the required paper format before the output starts, and @samp{-F} sends a 
FormFeed (ASCII 12) after the output.

For the HP Deskjet printer series, there is support of some ``special''
commands; @samp{-S 1} activates these. There is a limited support of color
modes available, too: For the DJ500C and newer models, @code{hp2xx} can
generate both CMY-based and CMYK-based color output (if colors are used:
see option @samp{-c}). Supply option @samp{-S 3} for CMY color mode, and
option @samp{-S 4} for CMYK color mode (for the DJ550C). The Deskjet modes
automatically invoke TIFF compression (mode 2), while pure PCL Level 3
does not know about compression.

A driver for the Esc/P2 raster language used by the Epson Stylus series
of inkjet printers has been added in @code{hp2xx-3.4.0}. This should work 
at least for the small-format printers at 720 dpi.

@node Preview, Misc. options, Printer formats, Basics
@comment  node-name,  next,  previous,  up
@section Preview

Preview mode is @code{hp2xx}'s default. Its use prior to all other
conversions is recommended since it offers a good impression of your
final results. Functionally it is also a raster mode.

Depending on your hardware and operating system platform,
@code{hp2xx} uses one of a variety of preview modules.
On GUIs, a window containing the graphics will pop up, while on other
systems the whole screen may be used for preview. You can control the
position of a preview window via options @samp{-o} and @samp{-O} in
a natural way. In full-screen previews, unused spaces are padded to the
right and bottom with background color.

Since there is no way for @code{hp2xx} to predict the actual size and
resolution of your preview device, e.g., screen, you may have to
gauge @code{hp2xx}'s preview mode (using options @samp{-whdD}). For example,
if your device effectively works at 86 DPI and offers an active area of
24 by 18 cm,
@example
hp2xx -d86 -w240 -h180 foo.hp
@end example
will make maximum use of your screen area and give you correct sizes.
Since a single gauge will do for all future calls, you'll probably want
to create some one-line batch file for invoking @code{hp2xx} in preview
mode, correctly gauged for your screen.


Depending on page offsets and the selected sizes and resolutions, a preview
may not fit on your screen. In that case, some preview modules simply
clip the picture; others give a warning but let you continue (DOS),
and others simply terminate --- so don't start too large.

DOS users: Most VGA cards offer high-resolution modes (SVGAs). Unfortunately,
there is no software standard for these modes. @code{hp2xx} lets you
utilize these modes anyway with just a little help from you.
Tell @code{hp2xx} the so-called mode byte of your favorite hi-res mode
via option @samp{-V @var{num}}. Since @code{hp2xx} issues only standard
BIOS calls for mode switching, setting of color look-up table entries,
and pixel drawing, chances are good that your VGA card's hi-res modes
will work!

WARNING: You can damage your hardware by specifying inappropriate VGA modes!
Generally you'll need a monitor which can sync on the horizontal frequency
of the selected VGA hi-res mode, e.g., a multi-scan monitor. In case of
doubt, switch off your monitor immediately!!



@node Misc. options, , Preview, Basics
@comment  node-name,  next,  previous,  up
@section Misc. options

@code{hp2xx} features an on-line options summary. Invoking @code{hp2xx}
with option @samp{-H}, or with any illegal option or without any
parameter, will display about 2 pages of text. (Note: I'd have preferred
option @samp{-h} for on-line help, but this option is needed by the
indispensable @var{height} parameter.)

During operation, @code{hp2xx} outputs various information about the current
HP-GL file and about @code{hp2xx}'s actions. As usual, all this goes to
@code{stderr}. You can re-direct these diagnostics into a file even without
any help from a UNIX shell by specifying a log file using option
@samp{-l @var{logfile}}, or you may switch off diagnostics completely
with option @samp{-q} (@samp{quiet} mode). NOTE: Using both options
as in @samp{-q -l @var{logfile}} is of no use as it will result in an
empty @var{logfile}.

Finally, there is a simple way to rotate whole pictures:
Option @samp{-r @var{angle}} rotates the picture counter-clockwise by
the supplied angle (given in degrees). E.g.,
@example
hp2xx -r90 foo.hp
@end example
will show the picture rotated by 90 degrees, letting vectors originally
pointing left-to-right now point botttom-to-top. This may be handy e.g. for
printing in landscape format.
NOTE: The limiting rectangle supplied by @samp{-hw} is not affected by
@samp{-r}, so in order to obtain e.g. a full-page landscape picture on an
A4 page, issue a command similar to:
@example
hp2xx -m pcl -d 150 -r90 -h270 -w160 landscape.hp
@end example

@node Advanced subjects, Installation and modification notes, Basics, Top
@comment  node-name,  next,  previous,  up
@chapter Advanced subjects

@menu
* The coordinate range:: What hp2xx tells you about your HP-GL file
* Fixed scaling::        Define your own window in HP-GL coordinate space
* Scaling to true size:: How to generate pictures in their original sizes
* Swapping::             Some suggestions which you hopefully won't need
* Dots and lines::       Simplifications and assumptions
* Unsupported formats::  The ATARI fraction's favourites
* TeX formats::          The pros and cons
@end menu

@node The coordinate range, Fixed scaling, , Advanced subjects
@comment  node-name,  next,  previous,  up
@section The coordinate range

The natural unit of length in HP-GL is 1/40 mm = 0.025 mm, so a typical
A4 page covers roughly 11000 x 7500 natural units. Typically, coordinates in
HP-GL commands will be found in the range 0 ... 12000. @code{hp2xx} will tell
you the maximum and minimum coordinates (``picture limits'')
it finds in your HP-GL picture for both x and y direction.
These values usually roughly cover this range.
Even if your HP-GL source plots in user-specific coordinates (realized
via HP-GL command @samp{SC;} (SCale) ), this remains true, since
@code{hp2xx} internally transforms all points back to natural coordinates.
Whenever the above range is grossly violated, you may suspect corrupted
data, because no real plotter would be able to plot such a file.

If you ever discover a picture limit equalling plus or minus 10^10,
your HP-GL probably didn't draw anything. Initially, @code{hp2xx}'s internal
picture limits are set to impossibly large (or small) values, i. e., +- 10^10,
but the first plot command will set them to values found therein, and successive
plots push the limits outward. Example: @var{xmax} starts at -10^10,
the first plot command may change it to 2536, the next to 3470, the next
20 command fall short, etc. Eventually, @var{xmax} assumes the largest
value and stays there. Knowledge about these details may sometimes
be crucial (@pxref{Scaling to true size}).

@code{hp2xx} uses the picture limits internally for scaling and fitting the
data into the supplied  limiting rectangle (@pxref{Sizing your output}).
You can also affect the picture limits yourself for special effects
(@pxref{Fixed scaling}).



@node Fixed scaling, Scaling to true size, The coordinate range, Advanced subjects
@comment  node-name,  next,  previous,  up
@section Fixed scaling

As noted earlier, @code{hp2xx} does not draw to scale, but rather it fits
a picture into a given limiting window. While this is very handy in most
applications, it may be undesirable when a series of pictures must be drawn
to the same scale. Unless all pictures possess the same picture limits
(modulo offsets), e.g., because all of them are surrounded by some fixed
frame, @code{hp2xx} would scale them all up differently to fit each of them
tightly into the limiting window.

There are two simple cures: First, make use of the true size option @samp{-t}.
If the original HP-GL sizes do not fit, adjust picture limits to
guarantee a constant scaling: Make a preview of all pictures and note
the coordinate ranges @code{hp2xx} reports. Then, determine picture limits
which cover all of these individual limits. Finally, run @code{hp2xx}
to create your desired outputs using options @samp{-xXyY} to tell
@code{hp2xx} about the picture limits it should use. If the pictures
do not share common offsets, you may have to correct for offsets
manually. Use the preview mode for testing. You'll get the same scale
as long as the limiting window and (@var{xmax} - @var{xmin}) and
(@var{ymax} - @var{ymin}) remain constant for all pictures.

WARNING: @code{hp2xx} does not clip lines. If the picture limits which
you manually can pre-set via options @samp{-xXyY} are chosen too narrow,
they will be pushed outside just as described in the last section,
resulting in a different scale. Check the coordinate ranges @code{hp2xx}
reports. The should match the values supplied by options @samp{-xXyY}!




@node Scaling to true size, Swapping, Fixed scaling, Advanced subjects
@comment  node-name,  next,  previous,  up
@section Scaling to true size

Earlier releases of @code{hp2xx} (binaries) did not offer option
@samp{-t}, which does everything you'll need for producing output with
exactly the sizes shown on a real plotter. The following paragraph
shows how to manually emulate the working of this option. Though outdated,
I left it in the manual as background material:

Sometimes you might want to create pictures sized exactly as if they
were drawn on a real plotter. There is a little trick which allows
you to do so using @code{hp2xx}: As notes above, the natural unit of
length in HP-GL is 0.025 mm. Therefore, you can calculate the true
picture size from the picture limits reported by @code{hp2xx}. Transform
these data into mm and simply specify the limiting window accordingly!
Example:
@samp{hp2xx truesize.hp} reports the following coordinate ranges:
@var{xmin} = 250, @var{xmax} = 5250, @var{ymin} = 100, @var{ymax} = 3100.
Thus, the picture is (@var{xmax} - @var{xmin}) * 0.025 mm = 125 mm wide
and @var{ymax} - @var{ymin}) * 0.025 mm = 75 mm high, and
@samp{hp2xx -w125 -h75 truesize.hp} will draw it in true size.




@node Swapping, Dots and lines, Scaling to true size, Advanced subjects
@comment  node-name,  next,  previous,  up
@section Swapping

@code{hp2xx} allocates memory for an internal bitmap dynamically.
Large pictures, high resolution, and use of colors may combine to
let your computer run out of memory (especially on non-swapping operating
systems like DOS).

In this case, @code{hp2xx} swaps the bitmap to disk, slowing down
considerably. Redirecting swapping to a fast disk, preferably a RAM disk,
might speed up things. You can replace the default swap file
@file{hp2xx.swp} using @samp{-s @file{swapfile}}.
NOTE: If for some reason @code{hp2xx} is aborted during swapping, you might
have to delete the swap file manually.





@node Dots and lines, Unsupported formats, Swapping, Advanced subjects
@comment  node-name,  next,  previous,  up
@section Dots and lines

Here are some basics about the generation of dots and lines within
@code{hp2xx}. I mention them, because there is something left to be
improved here...

Some HP-GL codes cause @code{hp2xx} to generate points rather than lines
of length zero. There is a subtle difference between both. Depending
on the current output format, special code for points will be generated,
and occasionally, a point will look different from a zero-length line.
Use @samp{-m epic} for such an example.

Line thicknesses can vary. Especially for thick lines, the matter of
line caps (how lines are ended, e.g. with a round cap) becomes relevant.
@code{hp2xx} does not do an elaborate job here. If line caps matter to you,
use @samp{-m eps}, edit the resulting Encapsulated PostScript file, look
for a line with @code{setlinecap} in it (near line 45), and select the
line cap of your choice by modifying the PostScript command @code{setlinecap}
accordingly. You can also use Metafont (via @samp{-m mf}) and replace
the picked pen "pencircle" by some other type. However, both methods
are far from convenient.

The internal rasterization done by @code{hp2xx} is a simple process
and may someday be replaced by something more efficient: A "draw point"
command essentially sets a single pel in the internal buffer.
If line width grows (2 - 4 units), a square of 2 to 4 pels length will
be set. Vector drawing is broken down to point drawing by the
Bresenham algorithm. Therefore, there is no notion of controlled
line caps. The shapes of line ends simply result from plotting these squares.
In addition, plotting all those pels is not really efficiently implemented,
so if anybody out there looks for a good place for speeding up @code{hp2xx},
this code (located in file @file{picbuf.c}) is a good place to start.

Currently there are no plans by me to introduce different line caps
into @code{hp2xx}, so waiting for them will be of no use.




@node Unsupported formats, TeX formats, Dots and lines, Advanced subjects
@comment  node-name,  next,  previous,  up
@section Unsupported formats

This is just a brief note, not a real manual entry -- sorry.
@table @code
@item PIC
	ATARI format, e.g. for the text processor Signum. Try to replace by IMG.

@item PAC
	ATARI format, e.g. for the CAD program STAD

@item DJ_GR
	DOS previewer, based on DJ Delorie's gcc port and extender go32.
	Works fine, but will be replaced by DOS/OS2 EMX version.
	Abandoned by new maintainer due to lack of platform - volunteers
	welcome.

@item OS2
	Full-screen OS/2 2.x and DOS previewer. HWW: I don't yet have the
	right development system, so this code is still missing.
	MK: Any volunteers ?

@item PM
	OS/2 2.0 PM previewer. Working, but without redirection of
	messages to stderr into a second window. Volunteers needed to
	keep OS/2 support up to date in new versions.
@end table




@node TeX formats, , Unsupported formats, Advanced subjects
@comment  node-name,  next,  previous,  up
@section @TeX{}@  formats

@TeX{}@  was designed for typesetting, not for handling graphics. Putting
graphics directly into @TeX{}@  therefore is always somewhat clumsy.
@code{hp2xx} offers four different compromises to do that, and much
better, though more indirect ways.

@samp{-m @code{mf}} generates Metafont source code. Run @code{Metafont}
and @code{gftopk}, and you'll end up with a special @code{pk} font
containing the single letter Z which represents your picture. Placing
this Z somewhere in your document using standard @TeX{}@  commands
draws your picture there.

If you want to avoid fiddling with additional programs and fonts, if you
work with La@TeX{}, and if you do not need high-quality plots,
the macros within @code{epic.sty} may help you.
@samp{-m @code{tex}} causes @code{hp2xx} to generate
appropriate @TeX{}@ source code which you can @samp{\input@{@}} into
La@TeX{}@ sources.

For em@TeX{}@ users, there are yet another two way: @samp{-m @code{em}}
creates @TeX{}@ code containing many commands like @samp{\special@{em:...@}}
for line drawing. The line drawing task will therefore be handled not by
@TeX{}@ itself but by the em@TeX{}@ drivers which can handle arbitrary
line slopes etc. Similarly, @samp{-m @code{cad}} produces code based on
the same principle, but compatible with program @file{@TeX{}cad.exe}, which is
distributed as a part of em@TeX{}, and which offers editing and drawing
features for the desired HP-GL figure(s).

Please note that all methods for generation of graphics @var{within}
@TeX{}@ are compromises which usually work only for simple graphics.
You'll probably prefer using external methods like including EPS vector
graphics files with Tom Rokicki's @code{dvips} driver, or PCX files via the
em@TeX{}@ drivers, or you'll generate special fonts with convenient
programs like F. Sowa's @code{bm2font}. @code{hp2xx} can help you in all
of these cases. The following table shows the pros and cons of the
various approaches (all are based on PD software):

@display
Internal methods (all allowing DVI previewing of graphs):

via Metafont
  @code{+:} Machine-independent; fully compatible with @TeX{}
  @code{-:} Slow; capacity problems with Metafont / gftopk / some DVI drivers
            if used with large and/or complex graphics

via @code{epic.sty}
  @code{+:} Machine-independent; single-step, native La@TeX{}@ approach; PD software
  @code{-:} Slow; requires La@TeX{}; low-quality lines; just one line thickness;
            complex graphs may exceed @TeX{}@ capacity

via em@TeX{}'s @code{\special@{em:...@}}
  @code{+:} No @TeX{}@ capacity problem; good line quality; single-step procedure;
            rasterization on demand, giving optimal resolution
  @code{-:} Slows down drivers; driver capacity may be exceeded; em@TeX{}@ required


External methods:

via PCX file inclusion:
  @code{+:} Easy and fast; DVI preview of graphics
  @code{-:} Requires em@TeX{}@ drivers (only available on DOS and OS/2)

via special fonts:
  @code{+:} Easy, fast, and trouble-free font generation via @code{bm2font};
            DVI preview of graphics (!); portable
  @code{-:} Many files for fonts etc.; confusing for novices

via EPS:
  @code{+:} High-quality results; easy; no burden for @TeX{}@ or drivers
  @code{-:} No DVI preview; PostScript printer (or, e.g., GhostScript) required;
            PostScript previewing is slower than DVI previewing.
@end display




@node Installation and modification notes, Appendix A, Advanced subjects, Top
@comment  node-name,  next,  previous,  up
@chapter Installation and modification notes

@menu
* Installation procedure::
* Modules of hp2xx::       The general structure, and how to add your own format
* Future improvements::    Volunteers wanted!
* Font coding::
@end menu

@node Installation procedure, Modules of hp2xx, , Installation and modification notes
@comment  node-name,  next,  previous,  up
@section Installation procedure

Please note: The following description is very brief and assumes that you
are familiar with installation of PD software in general.

@menu
* Installation of run-time versions::
* Source-level installation::
@end menu

@node Installation of run-time versions, Source-level installation, , Installation procedure
@comment  node-name,  next,  previous,  up
@subsection Installing an executable version

This is simple! If you find a collection of pre-compiled versions of
@code{hp2xx}, obtain the file @file{read.me} and read it to find out the
name of the file which fits to your system. Obtain it, rename it to something
like @file{hp2xx} or @file{hp2xx.exe}, and place it somewhere on your
search path -- that's it.

However, since the source release of @code{hp2xx} under the GNU public
license, the usual mode of installation has become configuring and 
building @code{hp2xx} on your own system. This is detailed in the next section. 



@node Source-level installation, , Installation of run-time versions, Installation procedure
@comment  node-name,  next,  previous,  up
@subsection Source-level installation

NOTE: I am tempted to switch to GNU autoconf , i.e. have a configure
script generate the necessary Makefile automagically. This will probably
not be welcomed by those not on Unix-like platforms, however.
Currently, installation depends too much
on manual work yet. Here is a description how to proceed:

After unbundling all sources, go to subdirectory @file{./makes}. Select a
makefile most closely resembling your system's needs from the samples given,
copy it to @file{./sources/makefile}, adapt it manually (if necessary),
and run @code{make all}.
If everything is set correctly, this results directly in a valid executable
file which you may install at any convenient place on your search path.

There are two types of makefile adaptation: First, let's assume there
is a makefile template available for your system. You then have the option to
add a few unsupported modes. Do so by un-commenting the appropriate lines
near the beginning of the makefile, and by commenting out the corresponding
standard lines.

The second type of course applies to systems with special needs which are not
yet covered by any makefile template. Currently, you are on your own when
it comes to supplying alternate paths, renaming or adding system libraries
and alike. Most probably you might have to tell the makefile where to
look for the X11 stuff, and where to find the png library, if at all.

Note: Don't feel alarmed if your makefile seems to neglect many source files.
Any single installation will make use of only one previewer (two on
SUNs with activated SunView support), and there are platform-dependent
sources for some output formats which are not always used.

Users wishing to compile @code{hp2xx} on MS Windows systems will find a
step-by-step procedure in the file @file{hp2xx_nt} in the documentation directory.


@node Modules of hp2xx, Future improvements, Installation procedure, Installation and modification notes
@comment  node-name,  next,  previous,  up
@section Adding your own formats

First, study @ref{Introduction} for the outline of the modular structure and
general operation of @code{hp2xx}.

Let's assume you want to support TIFF format. The probably easiest way
of adding new formats is by modifying copies of existing files. Since TIFF
is a raster format, a good starting point would be @file{to_pcx.c}.
(Files @file{to_mf.c} or @file{to_eps.c} should be considered in case
of a vector format, and @file{to_vga.c} or @file{to_x11.c} in case of
a new previewer.) Copy it to a file @file{to_tiff.c} and edit the latter.
The old code is pretty much self-explanatory. Essentially, the output file
is opened, initializations are performed, and the internal bitmap is
converted into the target format (here, TIFF) scanline-by-scanline.
There is just one routine called from other modules (originally named
@code{PicBuf_to_PCX}. Rename it to e.g. @code{PicBuf_to_TIFF} ) and adapt
the conversion code.

Once you've done that, the rest (integration of the new format into the
package) is easy: First, edit @file{hp2xx.h} and add a prototype
line for @code{PicBuf_to_TIFF} in analogy to e.g., @code{PicBuf_to_PCX}.
Edit the @file{makefile}s and add @code{to_tiff.c} to the list of sources
and e.g. @code{to_tiff.o} to the list of objects. Now you are ready for
compilation tests (but not for linking yet).

Then, change the main file @file{hp2xx.c} at various places: Near the
beginning of the file, add @file{XX_TIFF, } to the @code{hp2xx_mode} typedef,
and a line like @file{XX_TIFF,   "tiff", } to the @code{ModeList} struct below.
Please note the alphabetical order of these lists. Never put anything behind
the termination code @code{XX_TERM}! At the end of the file,
add a @code{case} statement to the @code{switch} list in analogy to e.g.
the @code{PCX} entry.

You may also want to add a line to the on-line help to
announce the new format, and change the release number and date.
Look for functions @code{Send_ID} and @code{usage_msg} at the first quarter
of file @file{hp2xx.c}!

Now a @code{make all} will produce code containing the new format.
If your format turns out to work nicely and seems to be of general interest,
please consider contributing it to the @code{hp2xx} project.



@node Future improvements, Font coding, Modules of hp2xx, Installation and modification notes
@comment  node-name,  next,  previous,  up
@section Future improvements

The following table lists miscellaneous desirable features for future
releases:
@display
  Box and sector drawing / filling
  Other, more rarely used HP-GL commands
  Color support in UIS 
  PCL: Better data compression for DJ500, DJ500C, DJ550;
  Loadable fonts, e.g. Hershey fonts, or: more built-in fonts
  Full-screen previewer for OS/2
  Easy installation on various platforms, e.g. via a configure script
@end display

@tex
\page
@end tex

@node Font coding, , Future improvements, Installation and modification notes
@comment  node-name,  next,  previous,  up
@section Font coding

This section is intended for those few users who might care to
improve the built-in character set of @code{hp2xx}.

HP-GL plotters feature built-in fonts with both fixed and
variable-width characters. There are commands for font selection
and quick switching between two pre-selected fonts, and there
is also a way for users to download own character definitions.

@code{hp2xx} currently features just a few fixed-width  character sets 
(sets 0-7).

If you plan to modify these characters set or to add more,
you need an understanding of how characters are drawn by
@code{hp2xx}. The source file @file{charset.h} contains a comment
explaining this procedure. Below you find a (modified) copy of
this:

@display
 This file defines a standard character set by elementary
 "draw" & "move" commands. The format is a very compact one from
 the old days where every byte was still appreciated.

 A font or character set is an array of strings. Each character is
 addressed by its ASCII code.

 A character is a (NULL-terminated) string of bytes. Each byte
 codes for a draw or move action according to the code below:

     @code{Bit: 7 6 5 4 3 2 1 0}
     @code{     p x x x y y y y}

 @code{p:   } Plot flag. If set, "draw to" new point, else "move to" it.
 @code{xxx: } 3-bit unsigned integer  (0...7). X coordinate of new point.
 @code{yyyy:} 4-bit unsigned integer (0..15). Y coordinate of new point.

 The baseline is y = 4 instead of y = 0, so characters with parts
 below it can be drawn properly. Function "code_to_ucoord" transforms
 these coordinates into actual user coordinates.

 Example:  code for character @code{'L': "\032\224\324"} translates to:
           @code{moveto(1,10); drawto(1,4); drawto(5,4);}

@tex
\page
@end tex

 From the example you can conclude that the font below essentially is
 defined on a 5x7 grid:

     @code{    0 1 2 3 4 5 6 7}
     @code{15  - - - - - - - -     - : unused}
     @code{14  - - - - - - - -     # : always used}
     @code{13  - - - - - - - -     o : sometimes used}
     @code{12  - - - - - - - -}
     @code{11  - - - - - - - -}
     @code{10  o # # # # # - -}
     @code{ 9  o # # # # # - -}
     @code{ 8  o # # # # # - -}
     @code{ 7  o # # # # # - -}
     @code{ 6  o # # # # # - -}
     @code{ 5  o # # # # # - -}
     @code{ 4  o # # # # # - -}
     @code{ 3  o o o o o o - -}
     @code{ 2  o o o o o o - -}
     @code{ 1  o o o o o o - -}
     @code{ 0  o o o o o o - -}
@end display



@node Appendix A, Appendix B, Installation and modification notes, Top
@comment  node-name,  next,  previous,  up
@appendix   Known HP-GL commands

@code{hp2xx} originally emulated a subset of the Hewlett-Packard 7550A plotter.
The following manual was used as reference for command definitions:
@cite{[1] HP 7550A Interfacing and Programming Manual}. This has gradually
been expanded to include allmost all of what is now known as HPGL/1, and
most of the commands from HPGL/2.

Not all commands are supported. Among the non-supported commands
are those which do not really apply to a software emulator, like:

@display
commands affecting the communication between plotter and host computer,
commands for changing the behaviour of a real plotter, like plotting speed etc.,
commands for the control of plotter memory allocation,
commands causing various plotter display outputs.
@end display

Other non-supported commands would be desirable, but were left out
due to their inherent complexity (or just because nobody had an immediate
need for them yet).

Programmers intending to add more HP-GL features should ideally take care to
implement the less-than-obvious side effects of existing commands on
the new features, too (and vice versa). E. g., line types (@code{LT;})
affect most but not all drawing commands: While the @code{ER;} command
(edge rectangle relative) uses the current line type, its counterpart
@code{EA;} (edge rectangle absolute) always draws solid lines. However,
both @code{PR;} and @code{PA;} use the current line type! In addition,
new features may need initializations by the already supported
codes @code{IN;} or @code{DF;}, so these may have to be expanded.
So carefully consult [1] prior to adding new HP-GL commands.

The remainder of this section lists all HP-GL commands given on
pages 1-2 to 1-4 of [1] and marks them as either

@display
      @code{(.)}  not applicable,
      @code{(-)}  ignored, or
      @code{(+)}  supported.
@end display

The label ``supported'' is used when I think the command is fully supported
in the context of the already implemented commands. In general,
you should have absolutely no problem with this class of commands.

Though there still are unsupported commands, this does not mean that
you might have trouble using @code{hp2xx}. Nowadays, most HP-GL files are
machine-generated, e.g. by CAD or DTP programs. These tend to make use
of just a simple subset of HP-GL. To my experience, chances are high
that @code{hp2xx} will give you the picture you want!

@display
@code{HP-GL|s n| Description & Remarks}
@code{Cmd  | i |}
@code{=====|===|========================================================}
@code{AA   |+  | Arc Absolute}
@code{AC   |+  | Anchor corner}
@code{AD   | - | Alternate font Definition}
@code{AF   |+  | Advance Full page [same as PG]}
@code{AH   |+  | Advance Half page [same as PG]}
@code{AP   |  .| Automatic pen operations}
@code{AR   |+  | Arc Relative}
@code{AS   |  .| Acceleration select}
@code{AT   |+  | Arc through three points}
@code{-----|---|--------------------------------------------------------}
@code{BF   | - | Buffer Plot [maybe not a valid hpgl command]}
@code{BL   |+  | Buffer Label}
@code{BP   |+  | Begin Plot [same as PG;IN]}
@code{BR   |+  | Bezier curve, Relative}
@code{BZ   |+  | Bezier curve, Absolute}
@code{-----|---|--------------------------------------------------------}
@code{CA   |+  | Designate alternate character set}
@code{CC   | - | Character Chord angle}
@code{CF   | - | Character Fill mode}
@code{CI   |+  | Circle}
@code{CM   |  .| Character selection mode}
@code{CO   |+  | File comment}
@code{CP   |+  | Character plot}
@code{CR   | - | Color Range}
@code{CS   |+  | Designate standard character set}
@code{CT   |+  | Chord tolerance}
@code{CV   |  .| Curved line generator}
@code{-----|---|--------------------------------------------------------}
@code{DC   |  .| Digitize clear}
@code{DF   |+  | Default}
@code{DI   |+  | Absolute direction}
@code{DL   | - | Define downloadable character}
@code{DP   |  .| Digitize point}
@code{DR   |+  | Relative direction}
@code{DS   | - | Designate character into slot}
@code{DT   |+  | Define label terminator}
@code{DV   |+  | text Direction Vertical}
@code{-----|---|--------------------------------------------------------}
@code{EA   |+  | Edge rectangle absolute}
@code{EC   |  .| Enable paper Cutter}
@code{EP   |+  | Edge polygon}
@code{ER   |+  | Edge rectangle relative}
@code{ES   |+  | Extra space}
@code{EW   |+  | Edge wedge}
@code{-----|---|--------------------------------------------------------}
@code{FI   | - | pcl Font ID}
@code{FN   | - | pcl secondary Font Number}
@code{FP   |+  | Fill polygon}
@code{FR   | - | FRame advance}
@code{FS   |  .| Force select}
@code{FT   |+  | Fill type}
@code{-----|---|--------------------------------------------------------}
@code{GC   |  .| Group count}
@code{GM   |  .| Graphics memory}
@code{-----|---|--------------------------------------------------------}
@code{IM   | - | Input error reporting mask}
@code{IN   |+  | Initialize}
@code{IP   |+  | Input P1 and P2}
@code{IR   | - | Input Relative P1 and P2}
@code{IV   | - | Invoke character slot}
@code{IW   |+  | Input window}
@code{-----|---|--------------------------------------------------------}
@code{KY   |  .| Define key}
@code{-----|---|--------------------------------------------------------}
@code{LA   | - | Line Attributes}
@code{LB   |+  | Label}
@code{LO   |+  | Label origin}
@code{LT   |+  | Line type}
@code{-----|---|--------------------------------------------------------}
@code{MC   | - | Merge Control}
@code{MG   |+  | Message [same as WD]}
@code{MT   | . | Media Type}
@code{-----|---|--------------------------------------------------------}
@code{NP   |+  | Number of Pens}
@code{NR   |  .| Not ready (unload page and go offline)}
@code{-----|---|--------------------------------------------------------}
@code{OA   |  .| Output actual position and pen status}
@code{OC   |  .| Output commanded position and pen status}
@code{OD   |  .| Output digitized point and pen status}
@code{OE   | - | Output error}
@code{OF   | - | Output factors}
@code{OG   |  .| Output group count}
@code{OH   | - | Output hard-clip limits}
@code{OI   |  .| Output identification}
@code{OK   |  .| Output key}
@code{OL   | - | Output label length}
@code{OO   |  .| Output options}
@code{OP   |+  | Output P1 and P2}
@code{OS   | - | Output status}
@code{OT   |  .| Output carousel type}
@code{OW   | - | Output window}
@code{-----|---|--------------------------------------------------------}
@code{PA   |+  | Plot absolute}
@code{PB   |+  | Print buffered label}
@code{PC   |+  | Pen Color}
@code{PD   |+  | Pen down}
@code{PE   |+  | Polyline Encoded}
@code{PG   |+  | Page feed}
@code{PR   |+  | Plot relative}
@code{PS   |+  | Plot Size}
@code{PT   |+  | Pen thickness}
@code{PU   |+  | Pen up}
@code{PW   |+  | Pen Width}
@code{-----|---|--------------------------------------------------------}
@code{QL   |  .| Quality Level} 
@code{-----|---|--------------------------------------------------------}
@code{RA   |+  | Fill rectangle absolute}
@code{RF   | - | Raster Fill pattern}
@code{RO   |+  | Rotate coordinate system}
@code{RP   | - | Replot}
@code{RR   |+  | Fill rectangle relative}
@code{RT   |+  | Relative arc through Three points}
@code{-----|---|--------------------------------------------------------}
@code{SA   |+  | Select alternate character set}
@code{SB   | - | Scalable or Bitmap font selection}
@code{SC   |+  | Scale}
@code{SD   | - | Standard font attribute Definition}
@code{SI   |+  | Absolute character size}
@code{SL   |+  | Character slant}
@code{SM   |+  | Symbol mode}
@code{SP   |+  | Select pen}
@code{SR   |+  | Relative character size}
@code{SS   |+  | Select standard character set}
@code{ST   |  .| Sort vectors}
@code{SV   | - | Screened Vectors}
@code{-----|---|--------------------------------------------------------}
@code{TD   | - | Transparent Data}
@code{TL   |+  | Tick length}
@code{TR   | - | Transparency mode}
@code{-----|---|--------------------------------------------------------}
@code{UC   |+  | User-defined character}
@code{UF   | - | User-defined fill type}
@code{UL   |+  | User-defined line type}
@code{-----|---|--------------------------------------------------------}
@code{VS   |  .| Velocity select}
@code{-----|---|--------------------------------------------------------}
@code{WD   |+  | Write to display}
@code{WG   |+  | Fill wedge}
@code{WU   |+  | pen Width Unit}
@code{-----|---|--------------------------------------------------------}
@code{XT   |+  | X-Tick}
@code{-----|---|--------------------------------------------------------}
@code{YT   |+  | Y-Tick}
@end display



@node Appendix B, Appendix C, Appendix A, Top
@comment  node-name,  next,  previous,  up
@appendix   Option summary

@menu
* General options::
* Size controls::
* Raster format controls::
* PCL specifics::
* Margins::
* Preview (DOS only)::
* Help::
@end menu

In the following, options are grouped into subjects and
are listed alphabetically within each subject. Both long options and
short (one-letter) options are listed, where short options appear in
parentheses. Except for the +DPI option, there is a one-to-one correspondence
between long and short options. You may use either long or short options.
Mixing long and short options is acceptable.

Option parameter names suggest the expected data type, e. g.,
@samp{--rotate (-r) @var{float}} means that option @samp{--rotate} or its
corresponding short form @samp{-r} expect a parameter of type @samp{float}.

@node General options, Size controls, , Appendix B
@comment  node-name,  next,  previous,  up
@section General options

@table @samp

@item --pencolors (-c) @var{string}
Pen color(s), a string of 1 to 8 digits.
Valid digits: 0...7 (0=Background or off, 1=Foreground, 2=Red, 3=Green,
4=Blue, 5=Cyan, 6=Magenta, 7=Yellow). Default: @samp{11111111}

@item --outfile (-f) @var{string}
Name of output file. If omitted, @code{hp2xx} generates the name from
the input file name and the current mode string. @samp{-f-} causes
@code{hp2xx} to write to @code{stdout}. Default: none.

@item --logfile (-l) @var{string}
Name of log file. If given, diagnostics go into this file, else to
@code{stderr}. Remark: @samp{-q} inhibits all diagnostics!

@item --mode (-m) @var{string}
Mode string. Valid: @var{string} =
@display
  @samp{cad}  (@TeX{}cad-compatible line generation using @code{\special@{em:...@}}),
  @samp{dxf}  (Autocad Drawing Exchange Format),
  @samp{em}   (more efficient line drawing with E. Mattes's @TeX{}@ @code{\special@{em:...@}}),
  @samp{eps}  (Encapsulated PostScript),
  @samp{gpt}  (Gnuplot input file),
  @samp{hpgl} (simplified HP-GL, useful for imports),
  @samp{img}  (e.g., GEM's Image format),
  @samp{jpg}  (JPEG image format),
  @samp{mf}   (Metafont source),
  @samp{pbm}  (Portable Bitmap (monochrome) or Portable Pixmap (color plots)),
  @samp{pcl}  (HP-PCL Level 3),
  @samp{pcx}  (Paintbrush format),
  @samp{pdf}  (Adobe Portable Document Format),
  @samp{png}  (Portable Network Graphics format),
  @samp{pre}  (Preview mode; no output!),
  @samp{rgip} (Uniplex RGIP vector format),
  @samp{svg}  (Scalable Vector Graphics, for XML web pages)
  @samp{tex}  (line drawing with @TeX{}@ / @code{epic} macros)
  @samp{fig}  (for Brian Smith's XFIG vector drawing program)
  @samp{tiff} (TIFF tagged image file format)
  @samp{esc2} (Epson Stylus printers)
@end display

Occasionally available (unsupported) modes:
@display
  @samp{ilbm} (e.g., for AMIGA: ILBM/IFF format),
  @samp{pac}  (e.g., for ATARI/STAD),
  @samp{pic}  (e.g., for ATARI/Signum).
@end display

Default: @samp{pre}.

@item --nofill (-n)
Ignore all polygon fill commands in the HPGL file(s). This may sometimes
be necessary to improve readability of images at small scales (e.g. thumbnails in
an overview listing), or to work around problems in the polygon fill code. 

@item --pensize (-p) @var{string}
Pensize(s), a string of 1 to 8 digits.
Valid digits: 0...9 (unit = 1/10 mm) for vector modes,
0...4 (unit = pixel) for raster modes. Default: @samp{11111111}

@item --pages (-P) @var{int}[:[@var{int}]]
Select HP-GL page @var{int} or a page range.
Valid: @var{int} integer and >= 0. Default: @var{int} = 0 (all pages).

@item --quiet (-q)
Quiet mode (no diagnostic output).

@item --rotation (-r) @var{float}
Rotation angle [deg]. E.g., @samp{-r90} gives landscape. Default: 0.0

@item --swapfile (-s) @var{string}
Name of swap file. Default: @var{string} = @samp{hp2xx.swp}.
@end table



@node Size controls, Raster format controls, General options, Appendix B
@comment  node-name,  next,  previous,  up
@section Size controls

@table @samp

@item --aspectfactor (-a) @var{float}
Aspect factor. Valid: @var{float} > 0.0. Use @var{float} > 1.0 for landscape
and @var{float} < 1.0 for portrait deformations. Default: @var{float} = 1.0

@item --height (-h) @var{float}
(Upper limit of) height of picture in mm. Default: @var{float}=200.0

@item --width (-w) @var{float}
(Upper limit of) width of picture in mm. Default: @var{float}=200.0

@item --x0 (-x) @var{float}
Pre-set left limit of HP-GL coordinate range to @var{float} (rarely used).

@item --x1 (-X) @var{float}
Pre-set right limit of HP-GL coordinate range to @var{float} (rarely used).

@item --y0 (-y) @var{float}
Pre-set lower limit of HP-GL coordinate range to @var{float} (rarely used).

@item --y1 (-Y) @var{float}
Pre-set upper limit of HP-GL coordinate range to @var{float} (rarely used).

@item --truesize (-t)
Ignore options @samp{-a -h -w} (aspect factor, height, width). Size information
will come from the HP-GL intrinsic data. WARNING: Avoid using option @samp{-r}
(rotate) as it works on top of HP-GL and thus will distort the detected HP-GL
sizes.

@end table



@node Raster format controls, PCL specifics, Size controls, Appendix B
@comment  node-name,  next,  previous,  up
@section  Raster format controls

@table @samp

@item --DPI (-d) @var{int}
Set x resolution to @var{int} dots per inch (DPI). If not overridden
by @samp{-D}, sets also y resolution to @var{int} DPI.
Valid: @var{int} integer and > 0. Default: @var{int} = 75.

@item --DPI_x (-d) @var{int}
Same as @samp{--DPI}

@item --DPI_y (-D) @var{int}
Set y resolution to @var{int} DPI. @var{int} integer and > 0.
Default: @var{int} = 75.
@end table



@node PCL specifics, Margins, Raster format controls, Appendix B
@comment  node-name,  next,  previous,  up
@section  PCL specifics

@table @samp

@item --PCL_formfeed (-F)
Send a FormFeed after graphics data. Default: No FormFeed.

@item --PCL_init (-i)
Pre-initialize printer. Default: No pre-init

@item --PCL_Deskjet (-S) @var{int}
Use (Deskjet) Special commands. @var{int} = 0 deactivates this option,
@var{int} = 1 enables b/w mode, @var{int} = 3 is intended for DJ500C
(CMY) color support, @var{int} = 4 supports DJ550C (CMYK mode).

@item --DPI_x (-d) @var{int}
Set x resolution (see above): Valid here: @var{int} = 75, 100, 150, 300

@item --DPI_y (-D) @var{int}
Set y resolution (see above). Invalid here!
@end table



@node Margins, Preview (DOS only), PCL specifics, Appendix B
@comment  node-name,  next,  previous,  up
@section  Margins

(Apply to modes @samp{eps}, @samp{pcl}, @samp{pre} ONLY)

@table @samp

@item --center (-C)
Center the picture within the frame defined by options @samp{-w -h} by
adding to the left or upper margin. Null effect if the width/height
ratio of the picture matches that of options @samp{-w -h}.

@item --no_ps (-N)
Ignore papersize definition given by a PS command in the HPGL file, and
calculate the actual image size. This may be necessary to avoid unnecessary
margins (with programs that write a standard header regardless of the true
size of the drawing).

@item --xoffset (-o) @var{float}
X offset of picture (left  margin) in mm.
Valid: @var{float} >= 0.0, default: @var{float}=0.0

@item --yoffset (-O) @var{float}
Y offset of picture (upper  margin) in mm.
Valid: @var{float} >= 0.0, default: @var{float}=0.0
@end table



@node Preview (DOS only), Help, Margins, Appendix B
@comment  node-name,  next,  previous,  up
@section  Preview (DOS/PC's only)

@table @samp

@item --VGAmodebyte (-V) @var{int}
VGA mode byte (decimal). Default: @var{int} = 18.
WARNING: Setting inappropriate VGA modes may damage your hardware,
especially your monitor!
@end table


@node Help, , Preview (DOS only), Appendix B
@comment  node-name,  next,  previous,  up
@section  Help
@table @samp
@item --help (-H)
(or calling @code{hp2xx} without any arguments)  Show on-line help.
@end table



@node Appendix C, , Appendix B, Top
@comment  node-name,  next,  previous,  up
@appendix

@menu
* Acknowledgement::
* Copyright note::
@end menu


@node Acknowledgement, Copyright note, , Appendix C
@comment  node-name,  next,  previous,  up
@section Acknowledgement

Since @code{hp2xx} first became publicly available (in its early days, as 
binaries on several platforms), many people contributed to this project by 
supplying first HWW and now MK with valuable suggestions, code patches and 
reports. Many thanks to all of them!
It is a pleasure to especially thank the following people for their
outstanding contributions:

HWW's Versions up to 3.2.0 (mid-1994):

@table @code

@item Nelson Beebe
Help with the new generic makefile (easier configuration);
clean selection mechanism for previewer suggested;
suggestions for an improved X.11 previewer

@item Elisabeth Dregger-Cappel
Network and host resources for original @code{hp2xx} distribution

@item Joern Eggers
New ATARI format "cs" for CS-TeX; bug fixes for arcs / circles

@item Roland Emmerich
DOS betatests and suggestions; @code{showit}

@item R. Frahm
PCX color improvements

@item Jonathan M. Gillian
DOS betatests and suggestions

@item Gilles Gravier
RGIP converter

@item Thomas Hiller
Code for EA support

@item Claus H. Langhans
AMIGA portation; pbm, ilbm formats

@item Lawrence Lowe
Many helpful suggestions and improvements, e.g. for pcl format

@item Ian MacPhedran
Color / binary format for pbm/ppm

@item Norbert Meyer
ATARI portation; img, pic, pac formats; first ATARI previewer

@item Michael Schmitz
Many VAX & MACH tests

@item Michael Schoene
X11 stuff; many tests

@item Andreas Schwab
Improved ATARI previewer

@item Friedhelm Sowa
Many DOS tests and suggestions for cooperation of @code{hp2xx} with
@TeX{} figure generation

@item Gerhard Steger
Many VAX tests; VAX scripts; access to MicroVAX platforms

@item Horst Szillat
OS/2 support, OS/2 full-screen previewer & help

@item Alois Treindl
Code for UC support


@item Versions since 3.3.0 (mid-1999):

@item Emmanuel Bigler
xfig and gnuplot ascii support

@item Ian MacPhedran
XFig output

@item Michael Schmitz
PNG output

@item Rolf Schreck
PS/RO fixes

@item Eugene Doudine
PE command support

@item Georgy Salnikov 
Improvements for character commands (LB,DI,DR)

@item Michele Liberi
TIFF output

@item Lars Erikson
EPS plotsize bug fixes

@item Gerhard Buergmann
bug fixes for plot scaling and PE support

@item James Shaw
polygon mode testing and examples, 
WinNT building instructions

@item Andrew Bird
enhanced PW support
TIFF enhancements
@end table



@node Copyright note, , Acknowledgement, Appendix C
@comment  node-name,  next,  previous,  up
@section Copyright notice

@display
Copyright (c) 1998 - 2000  Martin Kroeker
Copyright (c) 1991 - 1994  Heinz W. Werntges
All rights reserved.

Redistribution and use in source and binary forms are permitted provided that
the above copyright notice and this paragraph are duplicated in all such forms
and that any documentation, advertising materials, and other materials related
to such distribution and use acknowledge that the software was developed
by the abovementioned author(s).

THIS SOFTWARE IS PROVIDED ``AS IS'' AND WITHOUT ANY EXPRESS
OR IMPLIED WARRANTIES, INCLUDING, WITHOUT LIMITATION,
THE IMPLIED WARRANTIES OF MERCHANTIBILITY AND FITNESS
FOR A PARTICULAR PURPOSE.
@end display

@contents
@bye

